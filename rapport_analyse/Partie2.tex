
\chapter{Organisation du travail}


\section{Méthode de travail}

Nous avons cherché au mieux à répartir notre travail. Pour cela nous avons défini 3 grands axes de travail à l'issue de cette étude fonctionnelle.
\begin{itemize}
\item Dans un premier temps nous allons réaliser l'état de l'art.
\item Dans un deuxième temps nous étudierons la phase de réalisation.
\item Enfin nous testerons notre projet dans des conditions réelles.
\end{itemize}
~\\

Tout au long de ce projet nous avons choisi de réaliser notre travail en divisant notre équipe en 3 groupes de travail distincts formés respectivement de D'Acremont - Cotten, Legay - Rigaud, et Kenaan - Shehade. Notamment lors de l'état de l'art, ces groupes vont réaliser des recherches par binômes pour ensuite redistribuer les informations grâce aux outils mis à notre disposition (nous avons détaillé ces outils plus loin).
~\\

De plus, nous avons décidé lors de la phase de conception de diviser ce travail en plusieurs sous ensembles que nous définirons plus tard et qui seront chacun d'eux testés indépendemment, à l'image de tests unitaires en programmation.



\section{Outils utilisés}

Lors de notre projet nous avons choisi d'utiliser plusieurs outils de travail en collaboration.

\begin{itemize}
\item Nous utilisons Office 365. Nous avons créé un groupe de travail où nous partageons des fichiers et envoyons des mails de manière centralisée.
\item Nous utilisons également \LaTeX~pour la rédaction de nos rapports.
\item Nous pensons finalement utiliser Git et GitHub lors de notre phase de conception. Nous avons pour cela crée un projet sur GitHub.
\item Après plusieurs difficultés, nous avons réussi à utiliser Framaboard du groupe Framasoft pour gérer notre projet.
\end{itemize}








  


%%% Local Variables: 
%%% mode: latex
%%% TeX-master: "rapport_analyse"
%%% End: 
